%%%%%%%%%%%%%%%%%%%%%%%%%%%%%%%%%%%%%%%%%%%%%%%%%%%%%%%%%%%%%%%%%%%%%%%%%%%%%%%%%%%%%%%%%%%%%%%%%%%%%%%%%%
%
% Resumo
%
%%%%%%%%%%%%%%%%%%%%%%%%%%%%%%%%%%%%%%%%%%%%%%%%%%%%%%%%%%%%%%%%%%%%%%%%%%%%%%%%%%%%%%%%%%%%%%%%%%%%%%%%%%
\chapter*{Resumo}
\vspace{-2cm}

\noindent 
\vspace{0.5cm}

O experimento ν-Angra visa desenvolver uma nova técnica para monitorar a atividade de reatores nucleares por meio de um detector de antineutrinos de superfície. Como a energia térmica liberada na fissão é proporcional ao fluxo de antineutrinos, medindo-se a vazão dessas partículas, é possível inferir a quantidade de combustível que foi queimado e sua composição. Dada a complexidade do detector, simulações não necessárias para validação de eventos no detector. Este trabalho visa modificar a simulação atualmente usada através da alteração de alguns parâmetros com o intuito de aperfeiçoar a simulação atualmente utilizada.

\noindent Palavras-chave:  Simulação, Neutrinos, Geant4.\\

\newpage


