%%%%%%%%%%%%%%%%%%%%%%%%%%%%%%%%%%%%%%%%%%%%%%%%%%%%%%%%%%%%%%%%%%%%%%%%%%%%%%%%%%%%%%%%%%%%%%%%%%%%%%%%%%
%
% Resumo
%
%%%%%%%%%%%%%%%%%%%%%%%%%%%%%%%%%%%%%%%%%%%%%%%%%%%%%%%%%%%%%%%%%%%%%%%%%%%%%%%%%%%%%%%%%%%%%%%%%%%%%%%%%%
\chapter*{Abstract}
\vspace{-1.5cm}

\noindent 


The $\nu$-Angra experiment intends to devellop a new technique to monitor the activity of nuclear reactors trought a surface neutrino detector. As the released thermal energy of a fission reactor is proportional to the antineutrinos flux, measuring the flow rate of these particles allow us to infere the amount of burned combustive and its compositions. Given the complexity of the detector, a simulation is necessary to validate events in the detector. This work intendst to modify the now used simulation modifying some parameters to enhance the now used simulation.
\vspace{0.5cm}

\noindent Keywords: Simulation, Neutrinos, Geant4. \\

\newpage
