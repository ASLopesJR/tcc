\chapter{INTRODUÇÃO} \label{cap:intro}
\vspace{-2cm}



\section{Motivação} 

Para o experimento $\nu$-Angra, calibrar a simulação até que vá ao encontro dos dados reais é de essencial importância pois, com uma simulação que concorda com os dados aquistados, pode-se ajustar o sistema real com maior eficiência, além de realizar análises e prever eventos como o espectro do elétron de Michel e de um evento de antineutrino sem a interferência da saturação da eletrônica e a complexidade da resposta do detector à um evento. 

\section{Desenvolvimento}

Este trabalho tem o foco em estudar e aprimorar a simulação do experimento $\nu$-Angra realizada no \emph{Geant4}, levando em consideração a saturação dos \ac{PMTs}, da eletrônica de \emph{front-end} e da eletrônica de aquisição, e a qualidade da água do detector tendo em vista a aquisição de dados ocorrida de maio a julho de 2017.

\section{Mapa dos capítulos}

Dada a apresentação do tema neste capítulo, capítulo \ref{cap:experimento} apresenta o experimento $\nu$-Angra, dissertando sobre a motivação do experimento, o reator nuclear da usina Angra 2, especificações de montagem do detector e a eletrônica de aquisição e \emph{front-end}. O capítulo \ref{cap:dadosreais} apresenta uma análise dos dados de raios cósmicos aquisitados pelo detector enquanto posicionado ao lado do reator nuclear. O capítulo \ref{cap:simulacao} apresenta a plataforma \emph{Geant4} e os parâmetros utilizados pela simulação do experimento. O capítulo \ref{cap:resultados} mostra os resultados obtidos com a simulação e compara com os dados do experimento de fato. Por fim, o capítulo \ref{cap:conclusao} conclui o trabalho e mostra previsões futuras para o ajuste da simulação do experimento $\nu$-Angra.