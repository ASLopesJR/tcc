\chapter{Conclusão} \label{cap:conclusao}
\vspace{-2cm}

Este trabalho apresentou um estudo e aprimoramento da simulação do experimento $\nu$-Angra. O principal objetivo deste trabalho foi a comparação entre dados simulados e dados reais levando a algumas propostas de correção para o pacote de simulação relacionados à saturação do sistema de leitura do Experimento e ao parâmetro de simulação que ajusta a qualidade da água (caminho livre médio dos fótons). A simulação do experimento $\nu$-Angra ainda se encontra em fase de validação e uma busca por ajustes e/ou erros de implementação ainda devem ser considerados para que a mesma possa ser usado de modo mais conclusivo na processo de entendimento do detector e dos dados experimentais coletados.

\section{Próximos Passos}

Como indicado no Capítulo \ref{cap:simulacao}, a implementação da não-linearidade das PMTs e a saturação da eletrônica podem ser considerados na simulação caso haja interesse em se estudar eventos de mais altas energias (como considerado neste trabalho); a qualidade da água poderia ser medida diretamente e fontes radioativas poderiam ser usadas para fins de calibração do detector, para que se ajuste alguns parâmetros da simulação diminuindo ou até dando fim com a necessidade de se usar uma região mais energética dos eventos adquiridos pelo detector que saturam os canais da \textit{front-end} e das PMTs; pode-se também buscar uma calibração da água através da distribuição do elétron de Michel ao invés do espectro de múons uma vez que este anterior se encontra em uma região de mais baixa energia, com um impacto menor da saturação dos canais de leitura do detector na estimação das energias dos eventos. Deve-se também continuar com afinco a investigar sobre os motivos que levam às diferenças encontradas entre os dados reais e simulados para que seja possível encontrar eventuais erros de implementação, como aquela do vácuo nas PMTs, e, se possível, implementar um sistema \emph{multihreaded} para melhor eficiência computacional do projeto.