\chapter{Conclusão} \label{cap:conclusao}
\vspace{-2cm}

Este trabalho apresentou um estudo e aprimoramento da simulação do experimento $\nu$-Angra. O principal objetivo deste trabalho foi a correção de \emph{bugs} apresentados na simulação e apresentar um método para resolver o problema saturação da eletrônica e a não linearidade das PMTs na etapa de análise. A simulação do experimento $\nu$-Angra ainda apresenta muitas falhas e devido a complexidade da plataforma GEANT4, que devem ser abordadas em trabalhos futuros .

\section{Próximos Passos}

Como indicado no Capítulo \ref{cap:simulacao}, a não linearidade das PMTs e a saturação da eletrônica devem ser implementadas na simulação. Deve-se também implementar algumas modificações na estrutura da simulação como a correção de outros \emph{bugs} ainda não identificados e a implementação de um sistema \emph{multihreaded} para melhor eficiência computacional do projeto.